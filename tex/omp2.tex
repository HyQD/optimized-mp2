\documentclass[10pt,a4paper]{article}
\usepackage[utf8]{inputenc}
\usepackage{amsmath}
\usepackage{amsfonts}
\usepackage{amssymb}
\usepackage{braket}
\usepackage{dsfont}

%----------------------------------------------------------------------------------------
%       ACKNOWLEDGEMENTS PAGE DESIGN
%----------------------------------------------------------------------------------------

\newenvironment{acknowledgements}
{
  \thispagestyle{empty}
  \begin{center}
    {\huge{\textbf{Acknowledgements}} \par}
  \end{center}
}
{
  \vfil\vfil\vfil\null
  \thispagestyle{empty}
  }

\author{Håkon Emil Kristiansen}
\title{Optimized MP2}
\date{\today}

\begin{document}
\maketitle

\section{The OMP2 method}
The starting point of the OMP2 method is the NOCCD/OCCD Lagrangian 
\begin{equation}
\mathcal{H} = \braket{\Phi|(1+\hat{\Lambda}_2)e^{-\hat{T}_2}e^{-\hat{\kappa}}\hat{H}e^{\hat{\kappa}}e^{\hat{T_2}}|\Phi}
\end{equation}
where $(\hat{\Lambda}_2,\hat{T}_2)$ are cluster de-excitation and excitation operators and $\kappa$ is an orbital rotation operator defined by
\begin{equation}
\hat{\kappa} = \sum_{ai} \kappa^a_i a_a^\dagger a_i + \kappa^i_a a_i^\dagger a_a.
\end{equation}

Retaining only those terms giving up to second-order contributions to the Lagrangian defined the OMP2 Lagrangian
\begin{equation}
\mathcal{H}^{OMP2} = \braket{\Phi|\hat{H}|\Phi} + \braket{\Phi|\hat{\Lambda}_2\hat{H}|\Phi} + \braket{\Phi|[\hat{H},\hat{T}_2]|\Phi} + \braket{\Phi|\hat{\Lambda}_2[\hat{f}, \hat{T}_2 ]|\Phi}
\end{equation}

\section{OMP2 expressions}
We will find that the $\lambda$-amplitudes can be written in terms of the $\tau$-amplitudes and that $\kappa^i_a = -(\kappa^a_i)^*$.
\subsection{General/Unrestricted expressions}
In general we can write the one- and two-body density matrices as 
\begin{align}
\gamma_p^q &= (\gamma_{ref})_p^q+(\gamma_{corr})_p^q \\
\Gamma_{pr}^{qs} &= (\gamma_{ref})_{pr}^{qs}+(\gamma_{corr})_{pr}^{qs}. 
\end{align}
The reference terms are given by 
\begin{align}
(\gamma_{ref})_p^q &= \delta^q_j \delta^j_p \\
(\gamma_{ref})_{pr}^{qs} &= P(qs)P(pr) (\gamma_{corr})_p^q \delta^s_j \delta^j_r + P(qs)\delta^q_j\delta^j_p\delta^s_k\delta^k_r.
\end{align}

The non-zero elements of the one-body density matrix are given by 
\begin{align}
\gamma_i^j &= \delta^i_j - \frac{1}{2} \lambda^{jk}_{ab} \tau^{ab}_{ik} \\
\gamma_a^b &= \lambda^{ij}_{ac}\tau^{bc}_{ij}
\end{align}
and the non-zero elements of the two-body density are given by 
\begin{align}
\Gamma^{kl}_{ij} &= P(kl)P(ij) (\gamma_{corr})_i^k \delta^l_m \delta^m_j + P(kl)\delta^k_m\delta^m_i\delta^l_n\delta^n_j \\
\Gamma^{ab}_{ij} &= \tau^{ab}_{ij} \\
\Gamma^{ij}_{ab} &= \lambda^{ij}_{ab} \\
\Gamma^{jb}_{ia} &= \gamma^b_a \delta^j_k\delta^k_i \\
\Gamma^{bj}_{ia} &= -\gamma^b_a \delta^j_k\delta^k_i \\
\Gamma^{jb}_{ai} &= -\gamma^b_a \delta^j_k\delta^k_i \\
\Gamma^{bj}_{ai} &= \gamma^b_a \delta^j_k\delta^k_i. 
\end{align}

The equations for amplitudes and orbital rotations are given by 
\begin{align}
\frac{\partial \mathcal{H}^{OMP2}}{\partial \lambda^{ij}_{ab}} &= u^{ab}_{ij} - P(ij)f^k_j \tau^{ab}_{ik} + P(ab)f^a_c \tau^{cb}_{ij} \\
\frac{\partial \mathcal{H}^{OMP2}}{\partial \kappa^i_a} &= h^b_i \gamma^a_b - h^a_j \gamma^j_i + \frac{1}{2} \left( u^{pq}_{ir}\Gamma^{ar}_{pq}-u^{aq}_{rs}\Gamma^{rs}_{iq} \right),
\end{align}
where it is understood that $h^p_q, u^{pq}_{rs}$ are transformed/rotated matrix elements.

\subsection{Restricted/Closed-shell expressions}
We now assume closed-shell systems where each orbital is doubly occupied. Then the expression for the fock matrix is given by
\begin{equation}
f^p_q = h^p_q + 2u^{pj}_{qj}-u^{pq}_{jq}.
\end{equation}
The equations for amplitudes and orbital rotations are given by (using the biorthogonal paramterization of $\hat{\Lambda}, \hat{T}$ in Helgaker et. al) 
\begin{align}
\frac{\partial \mathcal{H}^{OMP2}}{\partial \lambda^{ij}_{ab}} &= u^{ab}_{ij} + P^{ab}_{ij} \left( f^a_c \tau^{cb}_{ij} - f^k_i \tau^{ab}_{kj} \right),  \\
\frac{\partial \mathcal{H}^{OMP2}}{\partial \kappa^i_a} &= h^b_i \gamma^a_b - h^a_j \gamma^j_i +  u^{pq}_{ir}\Gamma^{ar}_{pq}-u^{aq}_{rs}\Gamma^{rs}_{iq}.
\end{align}
One show that, 
\begin{equation}
\lambda^{ij}_{ab} = 2(2\tau^{ij}_{ab}-\tau^{ji}_{ab})^*.
\end{equation}

The expressions for the one- and two-body density matrices are given by 
\begin{align}
\gamma_i^j = 2\delta^j_i - \lambda^{kj}_{ab}\tau^{ab}_{ki} \\
\gamma^b_a = \lambda^{ij}_{ac} \tau^{bc}_{ij}
\end{align}
and 
\begin{align}
\Gamma^{kl}_{ij} &= 4\delta^k_i\delta^l_j - 2 \delta^l_i\delta^k_j + \hat{P}^{kl}_{ij} \left(-2\delta^k_i(\gamma_{corr})^l_j + \delta^k_j(\gamma_{corr})^l_i \right), \\
\Gamma^{ab}_{ij} &= 2(2\tau^{ab}_{ij}-\tau^{ab}_{ji}), \\
\Gamma^{ij}_{ab} &= \lambda^{ij}_{ab} = 2(2\tau^{ij}_{ab}-\tau^{ji}_{ab})^* = (\Gamma^{ab}_{ij})^*, \\
\Gamma^{jb}_{ia} &= 2\delta^j_i\gamma^b_a = \Gamma^{bj}_{ai}, \\
\Gamma^{bj}_{ia} &= -\delta^j_i \gamma^b_a = \Gamma^{jb}_{ai}.
\end{align}

\end{document}